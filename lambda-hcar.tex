\documentclass{article}

\usepackage{hcar}

\begin{document}

\begin{hcarentry}{Lambda Shell}
\report{Rob Dockins}
\status{beta, maintained}
\makeheader

The Lambda Shell is a feature-rich shell environment and command-line
tool for evaluating terms of the pure, untyped lambda calculus.  The Lambda
Shell builds on the shell creation framework Shellac, and showcases
most of Shellac's features.

Features of the Lamba Shell include:

\begin{itemize}
\item Evaluate lambda terms directly from the shell prompt using
      normal or applicative order.  In normal order, one can evaluate
      to normal form, head normal form, or weak head normal form.
\item Define aliases for lambda terms using a top level, non-recursive
      'let' construct.
\item Show traces of term evaluation, or dump the trace to a file.
\item Count the number of reductions when evaluating terms.
\item Test two lambda terms for beta-equivalence (that is; if two
      terms, when evaluated to normal form, are alpha equivalent).
\item Programs can be entered from the command line (using the -e option)
      or piped into stdin (using the -s option).
\item Perform continuation passing style (CPS) transforms on terms before
      evaluation using the double-bracket syntax, e.g.,  `[[ five ]]'.
\end{itemize}


The Lambda Shell was written as a showcase and textbook
example for how to use the Shellac shell-creation library.
However, it can also be used to gain a better understanding
of the pure lambda calculus.


\FurtherReading
\begin{itemize}
\item \url{http://www.cs.princeton.edu/~rdockins/lambda/home}
\item \url{http://www.cs.princeton.edu/~rdockins/shellac/home}
\end{itemize}
\end{hcarentry}

\end{document}
